Abbiamo visto come il polinomio $P(x) = x^3-4x^2+5x-2$, in $P(x)=0$ presenta due radici, una con molteplicità multipla $x=1$ e una con molteplicità semplice $x=2$.\\\\
Di seguito sono riportati tre codici MatLab, rispettivamente:
\begin{itemize}
\item \textbf{Metodo di Newton}
\lstinputlisting[language=Matlab]{Cap_2/Es_2/newton.m}
\item \textbf{Metodo di Newton modificato}
\lstinputlisting[language=Matlab]{Cap_2/Es_3/newtonMod.m}
\item \textbf{Metodo di Aitken}
\lstinputlisting[language=Matlab]{Cap_2/Es_3/aitken.m}
\end{itemize}
Il seguente codice MatLab, riguarda il polinomio $P(x) = x^3-4x^2+5x-2$, sul quale vengono eseguiti il metodo di Newton, il metodo di Newton modificato (con molteplicità $m=1$ per la radice $x=2$ e $m=2$ per la radice $x=1$) e il metodo di Aitken, valore di $tol_x=10^{-1}$ che decresce ad ogni passaggio, \textit{pd} che indica la derivata del polinomio, numero di iterazioni massime 1000 e punto di partenza $x_{0}=0$:\\
\lstinputlisting[language=Matlab]{Cap_2/Es_3/Es_3.m}
restituisce i seguenti valori:\\
\begin{center}
\begin{tabular}{|c|c|c|c|c|}
\hline
$tol_x$ & \textit{Newton} & \textit{Newton mod. m=1} & \textit{Newton mod. m=2} & \textit{Aitken} \\
\hline
$10^{-1}$ & $\tilde{x}$ = 0.8960 \quad \textit{i} = 4 & $\tilde{x}$ = 0.8960 \quad \textit{i} = 4 & $\tilde{x}$ = 0.9999 \quad \textit{i} = 3 & $\tilde{x}$ = 1.0020 \quad \textit{i} = 2\\
$10^{-2}$ & $\tilde{x}$ = 0.9929 \quad \textit{i} = 8 & $\tilde{x}$ = 0.9929 \quad \textit{i} = 8 & $\tilde{x}$ = 1.0000 \quad \textit{i} = 4 & $\tilde{x}$ = 1.0000 \quad \textit{i} = 3\\
$10^{-3}$ & $\tilde{x}$ = 0.9991 \quad \textit{i} = 11 & $\tilde{x}$ = 0.9991 \quad \textit{i} = 11 & $\tilde{x}$ = 1.0000 \quad \textit{i} = 4 & $\tilde{x}$ = 1.0000 \quad \textit{i} = 4\\
$10^{-4}$ & $\tilde{x}$ = 0.9999 \quad \textit{i} = 15 & $\tilde{x}$ = 0.9999 \quad \textit{i} = 15 & $\tilde{x}$ = 1.0000 \quad \textit{i} = 5 & $\tilde{x}$ = 1.0000 \quad \textit{i} = 4\\
$10^{-5}$ & $\tilde{x}$ = 1.0000 \quad \textit{i} = 18 & $\tilde{x}$ = 1.0000 \quad \textit{i} = 18 & $\tilde{x}$ = 1.0000 \quad \textit{i} = 5 & $\tilde{x}$ = 1.0000 \quad \textit{i} = 4\\
$10^{-6}$ & $\tilde{x}$ = 1.0000 \quad \textit{i} = 21 & $\tilde{x}$ = 1.0000 \quad \textit{i} = 21 & $\tilde{x}$ = 1.0000 \quad \textit{i} = 5 & $\tilde{x}$ = 1.0000 \quad \textit{i} = 4\\
$10^{-7}$ & $\tilde{x}$ = 1.0000 \quad \textit{i} = 25 & $\tilde{x}$ = 1.0000 \quad \textit{i} = 25 & $\tilde{x}$ = 1.0000 \quad \textit{i} = 5 & $\tilde{x}$ = 1.0000 \quad \textit{i} = 4\\
$10^{-8}$ & $\tilde{x}$ = 1.0000 \quad \textit{i} = 29 & $\tilde{x}$ = 1.0000 \quad \textit{i} = 29 & $\tilde{x}$ = 1.0000 \quad \textit{i} = 5 & $\tilde{x}$ = 1.0000 \quad \textit{i} = 4\\
$10^{-9}$ & $\tilde{x}$ = 1.0000 \quad \textit{i} = 29 & $\tilde{x}$ = 1.0000 \quad \textit{i} = 29 & $\tilde{x}$ = 1.0000 \quad \textit{i} = 5 & $\tilde{x}$ = 1.0000 \quad \textit{i} = 4\\
$10^{-10}$ & $\tilde{x}$ = 1.0000 \quad \textit{i} = 29 & $\tilde{x}$ = 1.0000 \quad \textit{i} = 29 & $\tilde{x}$ = 1.0000 \quad \textit{i} = 5 & $\tilde{x}$ = 1.0000 \quad \textit{i} = 4\\
$10^{-11}$ & $\tilde{x}$ = 1.0000 \quad \textit{i} = 29 & $\tilde{x}$ = 1.0000 \quad \textit{i} = 29 & $\tilde{x}$ = 1.0000 \quad \textit{i} = 5 & $\tilde{x}$ = 1.0000 \quad \textit{i} = 4\\
$10^{-12}$ & $\tilde{x}$ = 1.0000 \quad \textit{i} = 29 & $\tilde{x}$ = 1.0000 \quad \textit{i} = 29 & $\tilde{x}$ = 1.0000 \quad \textit{i} = 5 & $\tilde{x}$ = 1.0000 \quad \textit{i} = 4\\
$10^{-13}$ & $\tilde{x}$ = 1.0000 \quad \textit{i} = 29 & $\tilde{x}$ = 1.0000 \quad \textit{i} = 29 & $\tilde{x}$ = 1.0000 \quad \textit{i} = 5 & $\tilde{x}$ = 1.0000 \quad \textit{i} = 4\\
$10^{-14}$ & $\tilde{x}$ = 1.0000 \quad \textit{i} = 29 & $\tilde{x}$ = 1.0000 \quad \textit{i} = 29 & $\tilde{x}$ = 1.0000 \quad \textit{i} = 5 & $\tilde{x}$ = 1.0000 \quad \textit{i} = 4\\
$10^{-15}$ & $\tilde{x}$ = 1.0000 \quad \textit{i} = 29 & $\tilde{x}$ = 1.0000 \quad \textit{i} = 29 & $\tilde{x}$ = 1.0000 \quad \textit{i} = 5 & $\tilde{x}$ = 1.0000 \quad \textit{i} = 4\\
\hline
\end{tabular}
\end{center}
Si vede da questi risultati che i metodi di Newton modificato con molteplicità $m=2$ e di Aitken convergono molto velocemente alla soluzione, mentre il metodo di Newton e di Newton modificato con $m=1$ (tale valore di molteplicità rende identici i valori restituiti), seppur convergendo, richiedono più passi d'iterazione.