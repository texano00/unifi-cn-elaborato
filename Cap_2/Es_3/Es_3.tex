Abbiamo visto come il polinomio $P(x) = x^3-4x^2+5x-2$, in $P(x)=0$ presenta due radici, una con molteplicità multipla $x=1$ e una con molteplicità semplice $x=2$.\\\\
Di seguito sono riportati tre codici MatLab, rispettivamente:
\begin{itemize}
	\item \textbf{Metodo di Newton}
		\lstinputlisting[language=Matlab]{Cap_2/Es_2/newton.m}
	\item \textbf{Metodo di Newton modificato}
		\lstinputlisting[language=Matlab]{Cap_2/Es_3/newtonMod.m}
	\item \textbf{Metodo di Aitken}
		\lstinputlisting[language=Matlab]{Cap_2/Es_3/aitken.m}
\end{itemize}
Il seguente codice MatLab, riguarda il polinomio $P(x) = x^3-4x^2+5x-2$, sul quale vengono eseguiti il metodo di Newton, il metodo di Newton modificato (con molteplicità $m=1$ per la radice $x=2$ e $m=2$ per la radice $x=1$) e il metodo di Aitken, valore di $tol_x=10^{-1}$ che decresce ad ogni passaggio, \textit{pd} che indica la derivata del polinomio, numero di iterazioni massime 1000 e punto di partenza $x_{0}=0$:\\
	\lstinputlisting[language=Matlab]{Cap_2/Es_3/Es_3.m}
restituisce i seguenti valori:\\
\begin{center}
	\begin{tabular}{|c|c|c|c|c|}
		\hline
			$tol_x$ & \textit{Newton} & \textit{NewtonMod $m=2$} & \textit{Aitken} \\
		\hline
			$10^{-1}$ & $\tilde{x} = 0.895985715144219$ \quad $in = 4$ & $\tilde{x} = 0.999884326200135$ \quad $inm_2 = 3$ & $\tilde{x} = 1.00198731468863$ \quad $ia = 2$\\
			$10^{-2}$ & $\tilde{x} = 0.992894084175596$ \quad $in = 8$ & $\tilde{x} = 0.999999993312949$ \quad $inm_2 = 4$ & $\tilde{x} = 1.00000098931511$ \quad $ia = 3$\\
			$10^{-3}$ & $\tilde{x} = 0.999106262111969$ \quad $in = 11$ & $\tilde{x} = 0.999999993312949$ \quad $inm_2 = 4$ & $\tilde{x} = 0.999999998201194$ \quad $ia = 4$\\
			$10^{-4}$ & $\tilde{x} = 0.999944094597924$ \quad $in = 15$ & $\tilde{x} = 0.999999993312949$ \quad $inm_2 = 5$ & $\tilde{x} = 0.999999998201194$ \quad $ia = 4$\\
			$10^{-5}$ & $\tilde{x} = 0.999993011508562$ \quad $in = 18$ & $\tilde{x} = 0.999999993312949$ \quad $inm_2 = 5$ & $\tilde{x} = 0.999999998201194$ \quad $ia = 4$\\
			$10^{-6}$ & $\tilde{x} = 0.999999126508267$ \quad $in = 21$ & $\tilde{x} = 0.999999993312949$ \quad $inm_2 = 5$ & $\tilde{x} = 0.999999998201194$ \quad $ia = 4$\\
			$10^{-7}$ & $\tilde{x} = 0.999999945981418$ \quad $in = 25$ & $\tilde{x} = 0.999999993312949$ \quad $inm_2 = 5$ & $\tilde{x} = 0.999999998201194$ \quad $ia = 4$\\
			$10^{-8}$ & $\tilde{x} = 1.00000000137933$ \quad $in = 29$ & $\tilde{x} = 0.999999993312949$ \quad $inm_2 = 5$ & $\tilde{x} = 0.999999998201194$ \quad $ia = 4$\\
			$10^{-9}$ & $\tilde{x} = 1.00000000137933$ \quad $in = 29$ & $\tilde{x} = 0.999999993312949$ \quad $inm_2 = 5$ & $\tilde{x} = 0.999999998201194$ \quad $ia = 4$\\
			$10^{-10}$ & $\tilde{x} = 1.00000000137933$ \quad $in = 29$ & $\tilde{x} = 0.999999993312949$ \quad $inm_2 = 5$ & $\tilde{x} = 0.999999998201194$ \quad $ia = 4$\\
			$10^{-11}$ & $\tilde{x} = 1.00000000137933$ \quad $in = 29$ & $\tilde{x} = 0.999999993312949$ \quad $inm_2 = 5$ & $\tilde{x} = 0.999999998201194$ \quad $ia = 4$\\
			$10^{-12}$ & $\tilde{x} = 1.00000000137933$ \quad $in = 29$ & $\tilde{x} = 0.999999993312949$ \quad $inm_2 = 5$ & $\tilde{x} = 0.999999998201194$ \quad $ia = 4$\\
			$10^{-13}$ & $\tilde{x} = 1.00000000137933$ \quad $in = 29$ & $\tilde{x} = 0.999999993312949$ \quad $inm_2 = 5$ & $\tilde{x} = 0.999999998201194$ \quad $ia = 4$\\
			$10^{-14}$ & $\tilde{x} = 1.00000000137933$ \quad $in = 29$ & $\tilde{x} = 0.999999993312949$ \quad $inm_2 = 5$ & $\tilde{x} = 0.999999998201194$ \quad $ia = 4$\\
			$10^{-15}$ & $\tilde{x} = 1.00000000137933$ \quad $in = 29$ & $\tilde{x} = 0.999999993312949$ \quad $inm_2 = 5$ & $\tilde{x} = 0.999999998201194$ \quad $ia = 4$\\
		\hline
	\end{tabular}
\end{center}
Si vede da questi risultati che i metodi di Newton modificato con molteplicità $m=2$ e di Aitken convergono molto velocemente alla soluzione (arrotondando $\tilde{x}_{nm2}=0.999999993312949\approx1$ e $\tilde{x}_{a}=0.999999998201194\approx1$), mentre il metodo di Newton e di Newton modificato con $m=1$ (tale valore di molteplicità rende identici i valori restituiti, quindi non lo abbiamo calcolato), seppur convergendo (arrotondando $\tilde{x}_{n}=1.00000000137933\approx1$), richiedono più passi d'iterazione.