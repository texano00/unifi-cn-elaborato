Abbiamo visto come il polinomio $P(x) = x^3-4x^2+5x-2$, in $P(x)=0$ presenta due radici, una con molteplicità multipla $x=1$ e una con molteplicità semplice $x=2$.\\\\
Di seguito sono riportati tre codici MatLab, rispettivamente:
\begin{itemize}
	\item \textbf{Metodo di Newton}
		\lstinputlisting[language=Matlab]{Cap_2/Es_2/newton.m}
	\item \textbf{Metodo delle Corde}
		\lstinputlisting[language=Matlab]{Cap_2/Es_2/corde.m}
	\item \textbf{Metodo delle Secanti}
		\lstinputlisting[language=Matlab]{Cap_2/Es_2/secanti.m}
\end{itemize}
Il seguente codice MatLab, riguarda il polinomio $P(x) = x^3-4x^2+5x-2$, sul quale vengono eseguiti il metodo di Newton, il metodo delle Corde e il metodo delle Secanti (con secondo termine della successione ottenuto con Newton), valore di $tol_x=10^{-1}$ che decresce ad ogni passaggio, \textit{pd} che indica la derivata del polinomio, numero di iterazioni massime 1000 e punto di partenza $x_{0}=3$:\\
	\lstinputlisting[language=Matlab]{Cap_2/Es_2/Es_2.m}
restituisce i seguenti valori:\\
\begin{center}
	\begin{tabular}{|c|c|c|c|}
		\hline
			$tol_x$ & \textit{Newton} & \textit{Corde} & \textit{Secanti} \\
		\hline
			$10^{-1}$ & $\tilde{x} = 2.0043$ \quad $in = 4$ & $\tilde{x} = 2.2764$ \quad $ic = 3$ & $\tilde{x} = 2.1375$ \quad $is = 4$\\
			$10^{-2}$ & $\tilde{x} = 2.0000$ \quad $in = 5$ & $\tilde{x} = 2.0552$ \quad $ic = 12$ & $\tilde{x} = 2.1375$ \quad $is = 6$\\
			$10^{-3}$ & $\tilde{x} = 2.0000$ \quad $in = 6$ & $\tilde{x} = 2.0067$ \quad $ic = 27$ & $\tilde{x} = 2.0010$ \quad $is = 7$\\
			$10^{-4}$ & $\tilde{x} = 2.0000$ \quad $in = 6$ & $\tilde{x} = 2.0001$ \quad $ic = 44$ & $\tilde{x} = 2.0000$ \quad $is = 8$\\
			$10^{-5}$ & $\tilde{x} = 2.0000$ \quad $in = 7$ & $\tilde{x} = 2.0000$ \quad $ic = 62$ & $\tilde{x} = 2.0000$ \quad $is = 9$\\
			$10^{-6}$ & $\tilde{x} = 2.0000$ \quad $in = 7$ & $\tilde{x} = 2.0000$ \quad $ic = 79$ & $\tilde{x} = 2.0000$ \quad $is = 9$\\
			$10^{-7}$ & $\tilde{x} = 2.0000$ \quad $in = 7$ & $\tilde{x} = 2.0000$ \quad $ic = 96$ & $\tilde{x} = 2.0000$ \quad $is = 9$\\
			$10^{-8}$ & $\tilde{x} = 2.0000$ \quad $in = 7$ & $\tilde{x} = 2.0000$ \quad $ic = 113$ & $\tilde{x} = 2.0000$ \quad $is = 10$\\
			$10^{-9}$ & $\tilde{x} = 2.0000$ \quad $in = 8$ & $\tilde{x} = 2.0000$ \quad $ic = 131$ & $\tilde{x} = 2.0000$ \quad $is = 10$\\
			$10^{-10}$ & $\tilde{x} = 2.0000$ \quad $in = 8$ & $\tilde{x} = 2.0000$ \quad $ic = 148$ & $\tilde{x} = 2.0000$ \quad $is = 10$\\
			$10^{-11}$ & $\tilde{x} = 2.0000$ \quad $in = 8$ & $\tilde{x} = 2.0000$ \quad $ic = 165$ & $\tilde{x} = 2.0000$ \quad $is = 10$\\
			$10^{-12}$ & $\tilde{x} = 2.0000$ \quad $in = 8$ & $\tilde{x} = 2.0000$ \quad $ic = 182$ & $\tilde{x} = 2.0000$ \quad $is = 11$\\
			$10^{-13}$ & $\tilde{x} = 2.0000$ \quad $in = 8$ & $\tilde{x} = 2.0000$ \quad $ic = 199$ & $\tilde{x} = 2.0000$ \quad $is = 11$\\
			$10^{-14}$ & $\tilde{x} = 2.0000$ \quad $in = 8$ & $\tilde{x} = 2.0000$ \quad $ic = 217$ & $\tilde{x} = 2.0000$ \quad $is = 11$\\
			$10^{-15}$ & $\tilde{x} = 2.0000$ \quad $in = 8$ & $\tilde{x} = 2.0000$ \quad $ic = 233$ & $\tilde{x} = 2.0000$ \quad $is = 11$\\
		\hline
	\end{tabular}
\end{center}
Si vede da questi risultati che i metodi di Newton e delle Secanti convergono molto velocemente alla soluzione, mentre il metodo delle Corde, seppur convergendo, richiede molti più passi d'iterazione. Tuttavia, osservando il tempo d'esecuzione impiegato dai tre metodi per eseguire un singolo step, si deduce che i metodi quasi-Newton (Corde e Secanti) hanno un tempo di esecuzione medio per step inferiore a quello del metodo di Newton: infatti, in media, un passo d'iterazione del metodo delle secanti dura circa $\frac{1}{2}$ rispetto a quello di Newton e quello delle corde $\frac{1}{4}$. Quindi, in questo caso, il metodo più efficiente sembra essere quello delle secanti, che combina un'alta convergenza con un basso tempo di esecuzione.\\\\
La scelta del \textbf{valore di innesco} $x_{0}$ è importante. Un metodo \textit{converge localmente} ad $\alpha$ se la convergenza della successione dipende in modo critico dalla vicinanza di $x_{0}$ ad $\alpha$. Il procedimento è \textit{globalmente convergente} quando la convergenza non dipende da quanto $x_{0}$ è vicino ad $\alpha$. Per i metodi a convergenza locale la scelta del punto di innesco è cruciale.\\
E' possibile utilizzare $x_{0}=5/3$ come punto di innesco, in quanto essendo tutti e tre i metodi (\textbf{Newton, Corde e Secanti}) localmente convergenti, più la differenza con la radice è minore più velocemente converge.\\\\
Notiamo infatti che le distanze tra il nuovo punto di innesco $(x_0=\frac{5}{3})$ e le due radici del polinomio ($x_1=1$ e $x_2=2$) è in entrambi i casi minore rispetto alle distanze tra il vecchio punto di innesco $(x_0=3)$ e le stesse radici.
	\[
	|\alpha_1 - x_{0}| = |2 - 5/3| = 0,\overline{3} \leq 1 = |2 - 3| = |\alpha_1 - x|
	\]
	\[
	|\alpha_2 - x_{0}| = |1 - 5/3| = 0,\overline{6} \leq 2 = |1 - 3| = |\alpha_2 - x|
	\]
