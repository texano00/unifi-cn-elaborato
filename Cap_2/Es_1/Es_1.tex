Studio analitico del polinomio $P(x) = x^3-4x^2+5x-2$.\\
\begin{itemize}
	\item \textbf{Zeri del polinomio}\\
		Prima di tutto si scompone il polinomio :
			\[
			\ x^3-4x^2+5x-2 =
			\] 
			\[
			\ = x^3-2x^2+x-2-2x^2+4x =
			\]
			\[
			\ = x(x^2-2x+1)+2(1+x^2-2x) =
			\]
			\[
			\ = (x^2-2x+1)(x-2) =
			\]
			\[
			\ = (x-1)^2(x-2) 
			\]\\
		Quindi il polinomio si annulla $P(x)=0$ per $(x-1)=0 \Rightarrow x=1$ e $(x-2)=0 \Rightarrow x=2$. \\
	\item \textbf{Molteplicità}\\
		I valori di \textit{x} precedentemente calcolati vengono definiti come \textit{radici} del polinomio. Si dice che \textit{a} è una radice di \textit{P(x)} con \textit{molteplicità n} se e solo se \textit{P(x)} è divisibile per $(x-a)^n$, ma non è divisibile per $(x-a)^{n-1}$.\\
		Inoltre si dice che \textit{x} ha \textit{molteplicità esatta} $n \geq 1$, se:
			\[
			f(x) = f'(x) = ... = f^{(n-1)}(x) = 0,  f^{(n)}(x) \neq 0.
			\]
			\begin{itemize}
				\item $x=1$
					\[
					P(1) = 1-4+5-2 = 0 
					\]
					\[
					P'(1) = 3x^2-8x+5 = 3-8+5 = 0 
					\]
					\[
					P''(1) = 6x-8 = 6-8 \neq 0 \Rightarrow \textit{ molteplicità } n=2
					\]\\
				\item $x=2$
					\[
					P(2) = 8-16+10-2 = 0 
					\]
					\[
					P'(2) = 3x^2-8x+5 = 12-16+5 = 1 \neq 0 \Rightarrow \textit{ molteplicità } n=1
					\]
			\end{itemize}
		Quindi è che con $x=1$, la radice viene definita \textit{multipla} in quanto il polinomio viene annullato 2 volte, con molteplicità $n=2$; invece con $x=2$, la radice viene definita \textit{semplice} in quanto il polinomio viene annullato 1 volta,  con molteplicità $n=1$.\\ 
\end{itemize}
Il \textbf{metodo di bisezione} è utilizzabile per approssimarne uno delle due radice a partire dall'intervallo di confidenza \textit{[a,b]=[0,3]} se e solo se il polinomio dato $P(x)=0$ definito e continuo nell'intervallo di confidenza \textit{[a,b]=[0,3]}, tale che $P(a)*P(b) < 0$, è allora possibile calcolarne un'approssimazione in \textit{[a,b]}.\\\\
	\[
	P(a) = P(0) = -2
	\]
	\[
	P(b) = P(3) = 4
	\]
	\[
	P(a)*P(b) = -2 * 4 = -8 < 0
	\]\\
Essendo il polinomio continuo, le ipotesi sono rispettate. Infatti entrambe le radici $x \in \{1,2\}$, appartengono all'intervallo di confidenza \textit{[a,b]=[0,3]}.\\\\
Il seguente codice MatLab, riguarda il \textbf{Metodo di bisezione}:\\
	\lstinputlisting[language=Matlab]{Cap_2/Es_1/bisezione.m}
Il seguente codice MatLab, riguarda il polinomio $P(x) = x^3-4x^2+5x-2$, su quale viene eseguito il metodo di bisezione, con intervallo di confidenza \textit{[a,b]=[0,3]} e valore di $tol_x=10^{-1}$ che decresce ad ogni passaggio:\\
	\lstinputlisting[language=Matlab]{Cap_2/Es_1/Es_1.m}
restituisce i seguenti valori:\\
\begin{center}
	\begin{tabular}{|c|c|c|}
		\hline
			$tol_x$ & \textit{Bisezione} & \textit{Num. Iterazioni} \\
		\hline
   			$10^{-1}$ & $\tilde{x} = 1.500000000000000$ & $ib = 0$\\
    		$10^{-2}$ & $\tilde{x} = 1.992187500000000$ & $ib = 6$\\
    		$10^{-3}$ & $\tilde{x} = 2.000976562500000$ & $ib = 9$\\
    		$10^{-4}$ & $\tilde{x} = 2.000061035156250$ & $ib = 13$\\
   			$10^{-5}$ & $\tilde{x} = 1.999992370605469$ & $ib = 16$\\
   			$10^{-6}$ & $\tilde{x} = 2.000000953674316$ & $ib = 19$\\
    		$10^{-7}$ & $\tilde{x} = 2.000000059604645$ & $ib = 23$\\
    		$10^{-8}$ & $\tilde{x} = 1.999999992549419$ & $ib = 26$\\
    		$10^{-9}$ & $\tilde{x} = 2.000000000931323$ & $ib= 29$\\
    		$10^{-10}$ & $\tilde{x} = 2.000000000058208$ & $ib = 33$\\
    		$10^{-11}$ & $\tilde{x} = 1.999999999992724$ & $ib = 36$\\
    		$10^{-12}$ & $\tilde{x} = 2.000000000000910$ & $ib = 39$\\
    		$10^{-13}$ & $\tilde{x} = 2.000000000000057$ & $ib = 43$\\
    		$10^{-14}$ & $\tilde{x} = 1.999999999999993$ & $ib = 46$\\
    		$10^{-15}$ & $\tilde{x} = 2.000000000000001$ & $ib = 49$\\
		\hline
	\end{tabular}
\end{center}
Dalla tabella si può notare che la successione generata dal metodo di bisezione, a partile dall'intevallo [0,3], tende alla radice $x = 2$.