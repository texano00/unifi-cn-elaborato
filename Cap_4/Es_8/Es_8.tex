Il seguente codice MatLab contiene la soluzione del problema dell'Es.8, che consiste nel calcolo della \textit{costante di Lebesgue}, la cui formula è la seguente:\\\
	\[
		\Lambda_n = ||\lambda_n|| \quad \lambda_n(x) = \sum_{k=0}^{n} |L_{(k,n)}(x)|
	\]
	\lstinputlisting[language=Matlab]{Cap_4/Es_8/Es_8.m}
In tale codice viene utilizzata la funzione \textit{leb = lebesgue(x)}:\\\
	\lstinputlisting[language=Matlab]{Cap_4/Es_8/lebesgue.m}
Nella seguente tabella è riportato come varia la \textit{costante di Lebesgue} $\Lambda$, al variare del grado \textit{n} del polinomio e si può notare come la crescita sia \textit{ottimale}, per $n\rightarrow\infty$, prendendo in cosiderazione \textit{ascisse di Chebyshev}:\\\
	\begin{center}
		\begin{tabular}{|c|c|}
			\hline
				$n$ & $\Lambda$ \\
    		\hline
    			$2$  & $1.250000000000000$ \\ 
    			$4$  & $1.570160602269766$ \\ 
    			$6$  & $1.782456998086407$ \\ 
    			$8$  & $1.941512423650456$ \\ 
    			$10$ & $2.068187442224371$ \\ 
    			$12$ & $2.174110871180027$ \\ 
    			$14$ & $2.264722328850426$ \\ 
    			$16$ & $2.343983362402486$ \\ 
   				$18$ & $2.409451754790443$ \\ 
    			$20$ & $2.469143366927380$ \\ 
    			$22$ & $2.528774997717951$ \\ 
    			$24$ & $2.588559223241643$ \\ 
    			$26$ & $2.623067061433159$ \\ 
    			$28$ & $2.670035980466209$ \\ 
    			$30$ & $2.726634058782908$ \\ 
    			$32$ & $2.747846892506189$ \\ 
    			$34$ & $2.785213548019579$ \\ 
    			$36$ & $2.820923625604030$ \\ 
    			$38$ & $2.855383873068024$ \\ 
    			$40$ & $2.886802263254646$ \\ 
			\hline
		\end{tabular}
	\end{center}
