Il seguente codice MatLab contiene la soluzione del problema dell'Es.8, che consiste nel calcolo della \textit{costante di Lebesgue}, la cui formula è la seguente:\\\
	\[
		\Lambda_n = ||\lambda_n|| \quad \lambda_n(x) = \sum_{k=0}^{n} |L_{(k,n)}(x)|
	\]
	\lstinputlisting[language=Matlab]{Cap_4/Es_8/Es_8.m}
In tale codice viene utilizzata la funzione \textit{leb = lebesgue(x)}:\\\
	\lstinputlisting[language=Matlab]{Cap_4/Es_8/lebesgue.m}
che a sue volta richiama la funzione \textit{bl = polElemLagrange(z,x,i)}: \\\
	\lstinputlisting[language=Matlab]{Cap_4/Es_8/polElemLagrange.m}
Nella seguente tabella è riportato come varia la \textit{costante di Lebesgue} $\Lambda$, al variare del grado \textit{n} del polinomio e si può notare come la crescita sia \textit{ottimale}, per $n\rightarrow\infty$:\\\
	\begin{center}
		\begin{tabular}{|c|c|}
			\hline
				N & $\Lambda$ \\
    		\hline
    			$2$  & $$ \\ 
    			$4$  & $$ \\ 
    			$6$  & $$ \\ 
    			$8$  & $$ \\ 
    			$10$ & $$ \\ 
    			$12$ & $$ \\ 
    			$14$ & $$ \\ 
    			$16$ & $$ \\ 
   				$18$ & $$ \\ 
    			$20$ & $$ \\ 
    			$22$ & $$ \\ 
    			$24$ & $$ \\ 
    			$26$ & $$ \\ 
    			$28$ & $$ \\ 
    			$30$ & $$ \\ 
    			$32$ & $$ \\ 
    			$34$ & $$ \\ 
    			$36$ & $$ \\ 
    			$38$ & $$ \\ 
    			$40$ & $$ \\ 
			\hline
		\end{tabular}
	\end{center}
