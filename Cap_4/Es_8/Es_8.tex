Il seguente codice MatLab contiene la soluzione del problema dell'Es.8:
	\lstinputlisting[language=Matlab]{Cap_4/Es_8/Es_8.m}
Nella seguente tabella è riportato come varia la \textit{costante di Lebesgue} $\Lambda$
al variare del grado \textit{n} del polinomio e si può notare come la crescita sia \textit{ottimale}, per $n\rightarrow\infty$:\\\
	\begin{center}
		\begin{tabular}{|c|c|}
			\hline
				N & $\Lambda$ \\
    		\hline
    			$2$  & $0.441271200305303$ \\ 
    			$4$  & $0.882542400610606$ \\ 
    			$6$  & $1.140669505437423$ \\ 
    			$8$  & $1.323813600915910$ \\ 
    			$10$ & $1.465871197758856$ \\ 
    			$12$ & $1.581940705742726$ \\ 
    			$14$ & $1.680076076444663$ \\ 
    			$16$ & $1.765084801221213$ \\ 
   				$18$ & $1.840067810569542$ \\ 
    			$20$ & $1.907142398064159$ \\ 
    			$22$ & $1.967818743035501$ \\ 
    			$24$ & $2.023211906048029$ \\ 
    			$26$ & $2.074168676386841$ \\ 
    			$28$ & $2.121347276749966$ \\ 
    			$30$ & $2.165269502890975$ \\ 
    			$32$ & $2.206356001526516$ \\ 
    			$34$ & $2.244950834467166$ \\ 
    			$36$ & $2.281339010874846$ \\ 
    			$38$ & $2.315759272972476$ \\ 
    			$40$ & $2.315759272972476$ \\ 
			\hline
		\end{tabular}
	\end{center}
