Il problema relativo ad un moto rettilineo uniformemente accelerato, in forma polinomiale è :
	\[
		x(t) = x_0 + v_0t + a_0t^2 \quad con \quad a_0 = \frac{1}{2}a
	\]
il cui grado è $n=2$.\\\
Il problema è \textit{ben posto}, cioè ammette soluzione ed è unica, se e solo se almeno $n+1$ ascisse $x_i$ delle coppie dei dati, sono tra loro distinte.\\\
Nel nostro caso, abbiamo le seguenti coppie di dati $(tempo, spazio)=(x_i,y_i)$ per $i=0,...,n$:\\\
	\[
		(1,2.9),(1,3.1),(2,6.9),(2,7.1),(3,12.9),(3,13.1),(4,20.9),(4,21.1),(5,30.9),(5,31.1)	
	\]\\\
quindi $x_i=5$ ascisse distinte che sono $\geq$ di $n+1 = 2+1 = 3$, di conseguenza il problema risulta \textit{ben posto}.\\\
A questo punto possiamo stimare, nel senso dei \textit{minimi quadrati},posizione, velocità iniziale, ed accelerazione, che equivale alla risoluzione del sistema lineare determinato:\\\
	\[
		V\underline{a}=\underline{y}
	\]
	\[
		V=\begin{bmatrix}
			x_0^0 & x_0^1 & \cdots & x_0^m \\
			x_1^0 & x_1^1 & \cdots & x_1^m \\
			\vdots & \vdots & & \vdots \\
			x_n^0 & x_n^1 & \cdots & x_n^m \\		
		\end{bmatrix}
		\quad
		\underline{a}=\begin{bmatrix}
			a_0 \\
			a_1 \\
			\vdots \\
			a_m
		\end{bmatrix}
		\quad
		\underline{y}=\begin{bmatrix}
			y_0 \\
			y_1 \\
			\vdots \\
			y_n
		\end{bmatrix}
	\]\\\
in cui la matrice dei coefficienti $V \in \mathbb{R}^{n+1Xm+1}$ è una matrice di tipo \textit{Vandermonde} (in realtà la trasposta di una matrice di tipo \textit{Vandermonde}), il vettore \underline{a}, è il vettore  da determinare e definisce il polinomio di approssimazione ai \textit{minimi quadrati}, ed infine il vettore \underline{y} è il vettore dei \textit{valori misurati}.\\\
Quindi scambiando le incognite con i valori di Input abbiamo che :\\\
	\[
		V=\begin{bmatrix}
			1^0 & 1^1 & 1^2 \\
			1^0 & 1^1 & 1^2 \\
			2^0 & 2^1 & 2^2 \\
			2^0 & 2^1 & 2^2 \\
			3^0 & 3^1 & 3^2 \\
			3^0 & 3^1 & 3^2 \\
			4^0 & 4^1 & 4^2 \\
			4^0 & 4^1 & 4^2 \\
			5^0 & 5^1 & 5^2 \\
			5^0 & 5^1 & 5^2 				
		\end{bmatrix}
		=\begin{bmatrix}
			1 & 1 & 1 \\
			1 & 1 & 1 \\
			1 & 2 & 4 \\
			1 & 2 & 4 \\
			1 & 3 & 9 \\
			1 & 3 & 9 \\
			1 & 4 & 16 \\
			1 & 4 & 16 \\
			1 & 5 & 25 \\
			1 & 5 & 25 				
		\end{bmatrix} 
		\quad
		\underline{a}=\begin{bmatrix}
			x_0 \\
			v_0 \\
			a_0
		\end{bmatrix}
		\quad
		\underline{y}=\begin{bmatrix}
			2.9 \\
			3.1 \\
			6.9 \\
			7.1 \\
			12.9 \\
			13.1 \\
			20.9 \\
			21.1 \\
			30.9 \\
			31.1
		\end{bmatrix}
	\]\\\
ed il sistema lineare sovradeterminato da risolvere è :\\\
	\[
		\begin{bmatrix}
			1 & 1 & 1 \\
			1 & 1 & 1 \\
			1 & 2 & 4 \\
			1 & 2 & 4 \\
			1 & 3 & 9 \\
			1 & 3 & 9 \\
			1 & 4 & 16 \\
			1 & 4 & 16 \\
			1 & 5 & 25 \\
			1 & 5 & 25 				
		\end{bmatrix}
		\begin{bmatrix}
			x_0 \\
			v_0 \\
			a_0
		\end{bmatrix}=
		\begin{bmatrix}
			2.9 \\
			3.1 \\
			6.9 \\
			7.1 \\
			12.9 \\
			13.1 \\
			20.9 \\
			21.1 \\
			30.9 \\
			31.1
		\end{bmatrix}
	\]\\\
Tale sistema si risolve mediante fattorizzazione \textit{QR} (possibile poichè tutte le ascisse sono distinti).
Il seguente codice MatLab contiene la chiamata della funzione \textit{risolutoreQR} con Input la matrice $V$ e il vettore dei termini noti $b$:
	\lstinputlisting[language=Matlab]{Cap_4/Es_10/Es_10.m}
restituendo i seguenti risultati (vettore da determinare $\underline{a}$ e il vettore \textit{residuo} $\underline{r}$ e la rispettiva \textit{norma}):
	\[
		\underline{a}=\begin{bmatrix}
			1.000000000000001e+00 \\
			1.000000000000002e+00 \\
	 	    9.999999999999994e-01
		\end{bmatrix}
		\quad
		\underline{r}=\begin{bmatrix}
			1.000000000000023e-01 \\
   			-9.999999999999787e-02 \\
     		1.000000000000032e-01 \\
    		-9.999999999999609e-02 \\
     		1.000000000000014e-01 \\
    		-9.999999999999787e-02 \\
     		1.000000000000014e-01 \\
    		-1.000000000000014e-01 \\
     		9.999999999999787e-02 \\
    		-1.000000000000050e-01
		\end{bmatrix}
		\quad
		\|r\|_2^2 = 1.000000000000009e-01
	\]