\lstinputlisting[language=Matlab]{cap_3/es5/es5.m}

\begin{flushleft}
\textbf{Esempio risolutoreLDL^t}

Esempio: 
A = 
\begin{bmatrix}
	1 & 3 & 4 \\ 
	3 & 1 & -60 \\
	4 & -60 & -640 
\end{bmatrix}
x =
\begin{bmatrix}
	1 \\
	2 \\
	3 
\end{bmatrix}

Dove A = LDL^t

L = 
\begin{bmatrix}
	1 & 0 & 0 \\ 
	3 & 1 & 0 \\
	4 & 9 & 1 
\end{bmatrix}

D = 
\begin{bmatrix}
	1 & 0 & 0 \\ 
	0 & -8 & 0 \\
	0 & 0 & -8 
\end{bmatrix}


L^t = 
\begin{bmatrix}
	1 & 3 & 4 \\ 
	0 & 1 & 9 \\
	0 & 0 & 1 
\end{bmatrix}

Il risultato ottenuto di x è
\begin{bmatrix}
  -22.3750000000000 \\
  9.12500000000000 \\
  -1
\end{bmatrix}

\end{flushleft}
\begin{flushleft}
\textbf{Esempio risolutoreLUP}

Esempio: 
A = 
\begin{bmatrix}
	2 & 4 & 9 \\ 
	6 & 5 & 6 \\
	8 & 2 & 9 
\end{bmatrix}
xx =
\begin{bmatrix}
	1 \\
	2 \\
	3 
\end{bmatrix}

Dove A = LU
L = 
\begin{bmatrix}
	1 & 0 & 0 \\ 
	6 & 1 & 0 \\
	8 & 2 & 1 
\end{bmatrix}

U = 
\begin{bmatrix}
	2 & 4 & 9 \\ 
	0 & 5 & 6 \\
	0 & 0 & 9 
\end{bmatrix}

P = 
\begin{bmatrix}
	1 & 0 & 0 \\ 
	0 & 1 & 0 \\
	0 & 0 & 1 
\end{bmatrix}

Il risultato ottenuto di xx è
\begin{bmatrix}
  1.40000000000000 \\
  -1.20000000000000 \\
  0.333333333333333            
\end{bmatrix}

\end{flushleft}