Il seguente codice MatLab contiene la soluzione dell'$Es 4$:\\\
\lstinputlisting[language=Matlab]{Cap_6/Es_4/Es_4.m}
nel qualche viene richiamato:\\\
\begin{enumerate}
	\item \textbf{A = sparceMatrix(n)}\\\
		Effettua la creazione di \textit{matrici sparse} della dimensione $nXn$;\\\
	\item \textbf{y,i = gaussSeidel(A,b,tol,x0)}\\\
		Effettua sia il calcolo del \textit{vettore incognite} della matrice $A$, sia il numero di iterazioni impiegate, avendo come \textit{Input}, oltre la 	\textit{matrice}, il \textit{vettore termini noti}, e la \textit{tolleranza}, anche un \textit{vettore iniziale, in questo cado nullo}.\\\
		\lstinputlisting[language=Matlab]{Cap_6/Es_4/gaussSeidel.m}
	\begin{itemize}
		\item \textbf{u = mSolve(M,r)}\\\
			Effettua il calcolo di una matrice \textit{triangolare inferiore}.\\\
			\lstinputlisting[language=Matlab]{Cap_6/Es_4/mSolve.m}
	\end{itemize}
\end{enumerate}
e restituisce graficamente i seguenti risultati:\\\
