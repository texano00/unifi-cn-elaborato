Il seguente codice MatLab contiene la soluzione dell'$Es 5$:\\\
	\lstinputlisting[language=Matlab]{Cap_6/Es_5/Es_5.m}
nel qualche viene richiamato:\\\
	\begin{enumerate}
		\item \textbf{A = sparceMatrix(n)}\\\
			Effettua la creazione di \textit{matrici sparse} della dimensione $nXn$;\\\
		\item \textbf{x,i,B = jacobi(A,b,tol,x0)}\\\
			Effettua sia il calcolo del \textit{vettore incognite} della matrice $A$, sia il numero di iterazioni impiegate, sia il calcolo di una matrice $B$ contenente il \textit{passo di ogni iterazione} con il corrispettivo \textit{valore della norma}, avendo come \textit{Input}, oltre la \textit{matrice}, il \textit{vettore termini noti}, e la \textit{tolleranza}, anche un \textit{vettore iniziale, in questo caso nullo}.\\\
		\item \textbf{x,i,B = gaussSeidel(A,b,tol,x0)}\\\
			Effettua sia il calcolo del \textit{vettore incognite} della matrice $A$, sia il numero di iterazioni impiegate, sia il calcolo di una matrice $B$ contenente il \textit{passo di ogni iterazione} con il corrispettivo \textit{valore della norma}, avendo come \textit{Input}, oltre la 	\textit{matrice}, il \textit{vettore termini noti}, e la \textit{tolleranza}, anche un \textit{vettore iniziale, in questo caso nullo}.\\\
		\begin{itemize}
			\item \textbf{u = mSolve(M,r)}\\\
				Effettua il calcolo di una matrice \textit{triangolare inferiore}.\\\
		\end{itemize}
	\end{enumerate}
e restituisce graficamente i seguenti risultati:\\\
	\begin{figure}[H]
		\includegraphics[width=\textwidth]{Plot/Cap_6_Es_5}
	\end{figure}