Volendo conoscere quanto un errore influenzi il risultato quando $x\neq0$, si definisce l’errore relativo:
\[
|\epsilon_x| = \frac{\tilde{x}-x}{x} 
\]
da cui :
\[
\tilde{x} = x(1+\epsilon_x), \text{ e quindi } \frac{\tilde{x}}{x} = 1 + \epsilon_x
\]
ovvero l’errore relativo deve essere comparato a 1: un errore relativo vicino a zero indicherà che il risultato approssimato è molto vicino al risultato esatto,mentre un errore relativo uguale a 1 indicherà la totale perdita di informazione.\\\\
Con $x = e \approx 2.7183 = \tilde{x}, \text{l'errore relativo è quindi } |\epsilon_x| = \frac{2.7183 - e}{e} = 6.6849e-06$\\\\
Il numero di cifre significative \textit{k} corrette all’interno di $\tilde{x}$ si definisce con la formula :
\[
\textit{k} = -\log(2|\epsilon_x|)
\]
In questo caso il risultato del calcolo è $\textit{k}=4.8739$, che è abbastanza vicino alla realtà di $\textit{k}=5$ cifre significative corrette.\\\\
Spesso, per avere un’idea di quanto è l’ordine di grandezza di $\epsilon$ si scrive:
\[
|\epsilon_x| \approx \frac{1}{2}10^{-\textit{k}}
\]
infatti : 
\[
|\epsilon_x| \approx \frac{1}{2}10^{-4.8739} = 6.6849e-06 = |\epsilon_x|
\]