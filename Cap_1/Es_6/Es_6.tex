\begin{enumerate}
	\item
		Il seguente codice MatLab, riguarda la prima successione $x_{k+1} = (x_k + 3/x_k)/2$, indicando con $x=x_k$, $r=\epsilon$ e $conv = \sqrt{3} \approx 1.73205080756888e+000$ :\\
		\lstinputlisting[language=Matlab]{Cap_1/Es_6/FirstSucc.m}
		restituisce i seguenti valori:\\
		\begin{center}
			\begin{tabular}{|c|c|c|}
			\hline
				$k$ & $x_k$ & $\epsilon_k$ \\
			\hline
    			$k=0$ & $x_0 = 3.00000000000000$ & $\epsilon_0 = 1.267949192431123e+00$\\
    			$k=1$ & $x_1 = 2.00000000000000$ & $\epsilon_1 =  2.679491924311228e-01$\\
    			$k=2$ & $x_2 = 1.75000000000000$ & $\epsilon_2 = 1.794919243112281e-02$\\
    			$k=3$ & $x_3 = 1.732142857142857$ & $\epsilon_3 = 9.204957398001312e-05$\\
    			$k=4$ & $x_4 = 1.732050810014727$ & $\epsilon_4 = 2.445850189047860e-09$\\
    			$k=5$ & $x_5 = 1.732050807568877$ & $\epsilon_5 = 0.00000000000000e+000$\\
			\hline
			\end{tabular}
		\end{center}
		I calcoli indicano che per valori di $k$ superiori a 5, l'errore assoluto indicato con $\epsilon$, equivale a $0$, cioè $\leq$ \(10^{-12}\).\\
	\item
		Il seguente codice MatLab, riguarda la seconda successione $x_{k+1} = (3+x_{k-1}x_k)/(x_{k-1}x_k)$, indicando con $x=x_k$, $r=\epsilon$ e $conv = \sqrt{3} \approx 1.73205080756888e+000$ :\\
		\lstinputlisting[language=Matlab]{Cap_1/Es_6/SecondSucc.m}
		restituisce i valori:\\
		\begin{center}
			\begin{tabular}{|c|c|c|}
			\hline
				$k$ & $x_k$ & $\epsilon_k$ \\
			\hline
    			$k=0$ & $x_0 = 3.000000000000000$ & $\epsilon_0 = 1.267949192431123e+00$\\
    			$k=1$ & $x_1 = 2.000000000000000$ & $\epsilon_1 = 2.679491924311228e-01$\\
    			$k=2$ & $x_2 = 1.800000000000000$ & $\epsilon_2 = 6.794919243112285e-02$\\
    			$k=3$ & $x_3 = 1.736842105263158$ & $\epsilon_3 = 4.791297694280772e-03$\\
    			$k=4$ & $x_4 = 1.732142857142857$ & $\epsilon_4 = 9.204957397979108e-05$\\
    			$k=5$ & $x_5 = 1.732050934706042$ & $\epsilon_5 = 1.271371643518648e-07$\\
    			$k=6$ & $x_6 = 1.732050807572256$ & $\epsilon_6 = 3.378630708539276e-12$\\
    			$k=7$ & $x_7 = 1.732050807568877$ & $\epsilon_7 = 2.220446049250313e-16$\\
			\hline
			\end{tabular}
		\end{center}
		I calcoli indicano che per valori di $k$ superiori a 7 incluso, l'errore assoluto indicato con $\epsilon$, è dell'ordine di \(10^{-16}\), cioè $\leq$ \(10^{-12}\).\\
\end{enumerate}