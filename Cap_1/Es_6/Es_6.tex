\begin{enumerate}
\item
Il seguente codice MatLab, riguarda la prima successione $x_{k+1} = (x_k + 3/x_k)/2$, indicando con $x=x_k$, $r=\epsilon$ e $conv = \sqrt{3} \approx 1.73205080756888e+000$ :\\
\lstinputlisting[language=Matlab]{Cap_1/Es_6/FirstSucc.m}
restituisce i seguenti valori:\\
\begin{center}
\begin{tabular}{|c|c|c|}
\hline
$k$ & $x_k$ & $\epsilon$ \\
\hline
    0 & 3.00000000000000e+000 & 1.26794919243112e+000\\
    1 & 2.00000000000000e+000 & 267.949192431123e-003\\
    2 & 1.75000000000000e+000 & 17.9491924311228e-003\\
    3 & 1.73214285714286e+000 & 92.0495739800131e-006\\
    4 & 1.73205081001473e+000 & 2.44585018904786e-009\\
    5 & 1.73205080756888e+000 & 0.00000000000000e+000\\
\hline
\end{tabular}
\end{center}
I calcoli indicano che per valori di $k$ superiori a 4, l'errore assoluto indicato con $\epsilon$, è dell'ordine di \(10^{-9}\), cioè $\leq$ \(10^{-12}\).\\
\item
Il seguente codice MatLab, riguarda la seconda successione $x_{k+1} = (3+x_{k-1}x_k)/(x_{k-1}x_k)$, indicando con $x=x_k$, $r=\epsilon$ e $conv = \sqrt{3} \approx 1.73205080756888e+000$ :\\
\lstinputlisting[language=Matlab]{Cap_1/Es_6/SecondSucc.m}
restituisce i valori:\\
\begin{center}
\begin{tabular}{|c|c|c|}
\hline
$k$ & $x_k$ & $\epsilon$ \\
\hline
    0 & 3.00000000000000e+000 & 1.26794919243112e+000\\
    1 & 2.00000000000000e+000 & 267.949192431123e-003\\
    2 & 1.80000000000000e+000 & 67.9491924311229e-003\\
    3 & 1.73684210526316e+000 & 4.79129769428077e-003\\
    4 & 1.73214285714286e+000 & 92.0495739797911e-006\\
    5 & 1.73205093470604e+000 & 127.137164351865e-009\\
    6 & 1.73205080757226e+000 & 3.37863070853928e-012\\
    7 & 1.73205080756888e+000 & 222.044604925031e-018\\
\hline
\end{tabular}
\end{center}
I calcoli indicano che per valori di $k$ superiori a 6 incluso, l'errore assoluto indicato con $\epsilon$, è dell'ordine di \(10^{-12}\), cioè $\leq$ \(10^{-12}\).\\
\end{enumerate}