\begin{enumerate}
	\item
		Il seguente codice MatLab, riguarda la prima successione $x_{k+1} = (x_k + 3/x_k)/2$, indicando con $x=x_k$, $r=\epsilon$ e $conv = \sqrt{3} \approx 1.73205080756888e+000$ :\\
		\lstinputlisting[language=Matlab]{Cap_1/Es_6/FirstSucc.m}
		restituisce i seguenti valori:\\
		\begin{center}
			\begin{tabular}{|c|c|c|}
			\hline
				$k$ & $x_k$ & $\epsilon_k$ \\
			\hline
    			$k=0$ & $x_0 = 3.00000000000000e+000$ & $\epsilon_0 = 1.26794919243112e+000$\\
    			$k=1$ & $x_1 = 2.00000000000000e+000$ & $\epsilon_1 =  267.949192431123e-003$\\
    			$k=2$ & $x_2 = 1.75000000000000e+000$ & $\epsilon_2 = 17.9491924311228e-003$\\
    			$k=3$ & $x_3 = 1.73214285714286e+000$ & $\epsilon_3 = 92.0495739800131e-006$\\
    			$k=4$ & $x_4 = 1.73205081001473e+000$ & $\epsilon_4 = 2.44585018904786e-009$\\
    			$k=5$ & $x_5 = 1.73205080756888e+000$ & $\epsilon_5 = 0.00000000000000e+000$\\
			\hline
			\end{tabular}
		\end{center}
		I calcoli indicano che per valori di $k$ superiori a 4, l'errore assoluto indicato con $\epsilon$, è dell'ordine di \(10^{-9}\), cioè $\leq$ \(10^{-12}\).\\
	\item
		Il seguente codice MatLab, riguarda la seconda successione $x_{k+1} = (3+x_{k-1}x_k)/(x_{k-1}x_k)$, indicando con $x=x_k$, $r=\epsilon$ e $conv = \sqrt{3} \approx 1.73205080756888e+000$ :\\
		\lstinputlisting[language=Matlab]{Cap_1/Es_6/SecondSucc.m}
		restituisce i valori:\\
		\begin{center}
			\begin{tabular}{|c|c|c|}
			\hline
				$k$ & $x_k$ & $\epsilon_k$ \\
			\hline
    			$k=0$ & $x_0 = 3.00000000000000e+000$ & $\epsilon_0 = 1.26794919243112e+000$\\
    			$k=1$ & $x_1 = 2.00000000000000e+000$ & $\epsilon_1 = 267.949192431123e-003$\\
    			$k=2$ & $x_2 = 1.80000000000000e+000$ & $\epsilon_2 = 67.9491924311229e-003$\\
    			$k=3$ & $x_3 = 1.73684210526316e+000$ & $\epsilon_3 = 4.79129769428077e-003$\\
    			$k=4$ & $x_4 = 1.73214285714286e+000$ & $\epsilon_4 = 92.0495739797911e-006$\\
    			$k=5$ & $x_5 = 1.73205093470604e+000$ & $\epsilon_5 = 127.137164351865e-009$\\
    			$k=6$ & $x_6 = 1.73205080757226e+000$ & $\epsilon_6 = 3.37863070853928e-012$\\
    			$k=7$ & $x_7 = 1.73205080756888e+000$ & $\epsilon_7 = 222.044604925031e-018$\\
			\hline
			\end{tabular}
		\end{center}
		I calcoli indicano che per valori di $k$ superiori a 6 incluso, l'errore assoluto indicato con $\epsilon$, è dell'ordine di \(10^{-12}\), cioè $\leq$ \(10^{-12}\).\\
\end{enumerate}