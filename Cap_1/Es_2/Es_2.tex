Partiamo da
\[f'(x) \approx \frac{f(x+h) - f(x-h)}{2h}\]

Usando gli sviluppi di Taylor fino al secondo ordine otteniamo:
\[f(x+h) = f(x) + f'(x)h + \frac{1}{2} f''(x)h^{2} + \frac{1}{6}f'''(\xi_x)h^{3}\]
\[f(x-h) = f(x) - f'(x)h + \frac{1}{2} f''(x)h^{2} - \frac{1}{6}f'''(\mu_x)h^{3}\]

Al numeratore otteniamo
\[f(x+h) - f(x-h) = 2f'(x) + 2f'(x)h + \frac{1}{6}(f'''(\xi_x)h^{3}+f'''(\mu_x)h^{3})\]

La relazione iniziale diventa
\[f'(x) \approx \frac{f(x+h) - f(x-h)}{2h} - \frac{1}{12}(f'''(\xi_x) + f'''(\mu_x))h^{2}\]

Abbiamo quindi verificato, usando gli sviluppi di Taylor fino al secondo ordine con resto in forma di Lagrange, se f $\in C^{3}$ risulta
\[f'(x) = \phi_h(x) + O(h^2)\]
dove
\[\phi_h(x) = \frac{f(x+h) - f(x-h)}{2h}\]