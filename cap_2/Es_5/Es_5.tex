Come precedentemente visto nell'Esercizio 2.4 si utilizzerà la funzione $f(x) = x^2-\alpha$, che ha radici semplici in $x=\sqrt{\alpha}$ e in $x=-\sqrt{\alpha}$, ovvero $f(\pm\sqrt{\alpha})=0$. La derivata prima di questa funzione è $f'(x)=2x$.\\\\
L'iterazione del metodo delle Secanti utilizzando questa funzione diventa :
\[
x_{i+1} = \frac{f(x_i)x_{i-1}-f(x_{i-1})x_i}{f(x_i)-f(x_{i-1})} =
\]
\[
= \frac{(x_i^2-\alpha)x_{i-1}-(x_{i-1}^2-\alpha)x_i}{x_i^2-\alpha-x_{i-1}^2+\alpha}  =
\]
\[
= \frac{x_i^2x_{i-1}-\alpha x_{i-1}-x_{i-1}^2x_i+\alpha x_i}{x_i^2-x_{i-1}^2} =
\]
\[
= \frac{x_ix_{i-1}(x_i-x_{i-1})+\alpha (x_i-x_{i-1})}{(x_i-x_{i-1})(x_i+x_{i-1})} =
\]
\[
= \frac{(x_i-x_{i-1})(x_ix_{i-1}+\alpha)}{(x_i-x_{i-1})(x_i+x_{i-1})} =
\]
\[
= \frac{x_ix_{i-1}+\alpha}{x_i+x_{i-1}},\quad i=0,1,2,...
\]\\\\
Il seguente codice MatLab, riguarda l'implementazione del \textbf{metodo delle Secanti per il calcolo} $\sqrt{\alpha}$:\\ 
\lstinputlisting[language=Matlab]{Cap_2/Es_5/secantisqrtalpha.m}
Il seguente codice MatLab, riguarda la chiamata della funzione definita precedentemente, con $\alpha=x_0=5$, con $x_1=3$, con numero di passi massimi $imax=100$ e indice di tolleranza $tol_x=eps$ :\\
\lstinputlisting[language=Matlab]{Cap_2/Es_5/Es_5.m}
restituisce i seguenti valori:\\
\begin{center}
\begin{tabular}{|c|c|c|}
\hline
$i$ & $x_i$ & $E_{ass}=\epsilon=|x_i-\sqrt{\alpha}| \quad \alpha=5$ \\
\hline
    0 & $x_0$ = 5 & $|\epsilon|$ = 2.763932022500210e+00\\
    1 & $x_1$ = 3 & $|\epsilon|$ = 7.639320225002102e-01\\
    2 & $x_2$ = 2.500000000000000e+00 & $|\epsilon|$ = 2.639320225002102e-01\\
    3 & $x_3$ = 2.272727272727273e+00 & $|\epsilon|$ = 3.665929522748312e-02\\
    4 & $x_4$ = 2.238095238095238e+00 & $|\epsilon|$ = 2.027260595448332e-03\\
    5 & $x_5$ = 2.236084452975048e+00 & $|\epsilon|$ = 1.647547525829296e-05\\
    6 & $x_6$ = 2.236067984964863e+00 & $|\epsilon|$ = 7.465073448287285e-09\\
    7 & $x_7$ = 2.236067977499817e+00 & $|\epsilon|$ = 2.753353101070388e-14\\
    8 & $x_8$ = 2.236067977499790e+00 & $|\epsilon|$ = 4.440892098500626e-16\\
    9 & $x_9$ = 2.236067977499790e+00 & $|\epsilon|$ = 0\\
    10 & $x_{10}$ = 2.236067977499790e+00 & $|\epsilon|$ = 0\\
\hline
\end{tabular}
\end{center}
Si può notare come la \textit{convergenza superlineare} sia leggermente più lenta rispetto alla \textit{convergenza quadratica} del metodo di Newton visto nell'esercizio precedente (2.4).