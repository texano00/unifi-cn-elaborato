Essendo $\sqrt{\alpha}$ la radice ricercata, dobbiamo innanzitutto trovare una funzione $f(x)$ che abbia uno zero in $x=\sqrt{\alpha}$. La funzione più semplice di questo tipo è $f(x)=x-\sqrt{\alpha}$, ma ovviamente, dato che si sta tentando di approssimare $\sqrt{\alpha}$ stessa, non è verosimile utilizzare il valore esatto per il calcolo dell'approssimazione. Quindi si
utilizza la funzione $f(x) = x^2-\alpha$, che ha radici semplici in $x=\sqrt{\alpha}$ e in $x=-\sqrt{\alpha}$, ovvero $f(\pm\sqrt{\alpha})=0$. La derivata prima di questa funzione è $f'(x)=2x$.\\\\
L'iterazione del metodo di Newton utilizzando questa funzione diventa :
\[
x_{i+1} = x_i-\frac{f(x_i)}{f'(x_i)} = x_i - \frac{x_i^2-\alpha}{2x_i} =
\]
\[
= \frac{2x_i^2-x_i^2+\alpha}{2x_i} = \frac{x_i^2+\alpha}{2x_i} =
\]
\[
= \frac{1}{2} \Bigl( x_i+\frac{\alpha}{x_i} \Bigl),\quad i=0,1,2,...
\]\\\\
Il seguente codice MatLab, riguarda l'implementazione del \textbf{metodo di Newton per il calcolo} $\sqrt{\alpha}$:\\ 
\lstinputlisting[language=Matlab]{Cap_2/Es_4/newtonsqrtalpha.m}
Il seguente codice MatLab, riguarda la chiamata della funzione definita precedentemente, con $\alpha=x_0=5$, con numero di passi massimi $imax=100$ e indice di tolleranza $tol_x=eps$ :\\
\lstinputlisting[language=Matlab]{Cap_2/Es_4/Es_4.m}
restituisce i seguenti valori:\\
\begin{center}
\begin{tabular}{|c|c|c|}
\hline
$i$ & $x_i$ & $E_{ass}=\epsilon=|x_i-\sqrt{\alpha}| \quad \alpha=5$ \\
\hline
    0 & $x_0$ = 5 & $|\epsilon|$ = 2.763932022500210e+00\\
    1 & $x_1$ = 3 & $|\epsilon|$ = 7.639320225002102e-01\\
    2 & $x_2$ = 2.333333333333333e+00 & $|\epsilon|$ = 9.726535583354368e-02\\
    3 & $x_3$ = 2.238095238095238e+00 & $|\epsilon|$ = 2.027260595448332e-03\\
    4 & $x_4$ = 2.236068895643363e+00 & $|\epsilon|$ = 9.181435736138610e-07\\
    5 & $x_5$ = 2.236067977499978e+00 & $|\epsilon|$ = 1.882938249764265e-13\\
    6 & $x_6$ = 2.236067977499790e+00 & $|\epsilon|$ = 0\\
    7 & $x_7$ = 2.236067977499790e+00 & $|\epsilon|$ = 0\\
\hline
\end{tabular}
\end{center}