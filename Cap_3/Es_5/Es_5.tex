Il seguente codice MatLab, contiene la chiamata delle due funzioni descritte negli esercizi precedenti, (\textit{risolutoreLDLt}  dell'es. 3.3 e risolutoreLUpiv dell'es. 3.4) per dimostrarne l'utilizzo tramite alcuni esempi:\\\
\lstinputlisting[language=Matlab]{Cap_3/Es_5/Es_5.m}
\begin{description}
\item \textbf{Esempio :} \textit{$risolutoreLDL^t$}\\\\
	con i seguenti parametri di input :\\\	 
	\[
	A = LDL^t =\begin{bmatrix}
		 3 & 2 & -1 \\ 
		 2 & 7 &  7 \\
		-1 & 7 & 30 
	\end{bmatrix} \quad
	b =\begin{bmatrix}
		1 \\
		2 \\
		3                
	\end{bmatrix}
	\]\\
	\[
	L =\begin{bmatrix}
		   1   &   0    & 0 \\ 
		0,6667 &   1    & 0 \\
	   -0,3333 & 1.3529 & 1 
	\end{bmatrix} \quad
	D =\begin{bmatrix}
		3 &   0    &   0    \\ 
		0 & 5.6667 &   0    \\
		0 &   0    & 19.2941 
	\end{bmatrix} \quad
	L^t =\begin{bmatrix}
		1 & 0.6667 & -0.3333 \\ 
		0 &    1   &  1.3529 \\
		0 &    0   &    1 
	\end{bmatrix}
	\]\\\\
	Il risultato ottenuto è:\\
	\[
	x =\begin{bmatrix}
		0.2744 \\
		0.1280 \\
		0.0793                
	\end{bmatrix}
	\]\\\
\item \textbf{Esempio :} \textit{$risolutoreLUP$}\\\\	
	con i seguenti parametri di input :\\\	  
	\[
	A = LU =\begin{bmatrix}
		1  & 3 & 2 \\ 
		-1 & 2 & 1 \\
		4  & 1 & 2 
	\end{bmatrix} \quad
	b =\begin{bmatrix}
		1 \\
		2 \\
		3 
	\end{bmatrix}
	\]\\
	\[
	L =\begin{bmatrix}
		   1   &  0  & 0 \\ 
		0.3333 &  1  & 0 \\
		0.6667 & 0.2 & 1 
	\end{bmatrix} \quad
	U =\begin{bmatrix}
		3 &    2    &    1   \\ 
		0 & -1.6667 & 3.6667 \\
		0 &    0    &   0.6 
	\end{bmatrix}
	\]\\ 
	\[
	P =\begin{bmatrix}
		0 & 1 & 0 \\ 
		1 & 0 & 0 \\
		0 & 0 & 1 
	\end{bmatrix} \quad
	Pb =\begin{bmatrix}
		2 \\
		1 \\
		3                
	\end{bmatrix}
	\]\\\\
	Il risultato ottenuto è:\\
	\[
	xx =\begin{bmatrix}
		  -4   \\
		5.6667 \\
		2.6667 
	\end{bmatrix}
	\]\\\\
\end{description}
La tabella sottostante contiene, per ogni esempio considerato, il numero di condizionamento di \textit{A}, in \textit{norma 2}, con l'utilizzo del comando \textit{cond} di Matlab :\\\
\begin{center}
\begin{tabular}{ | l | c | r | }
	\hline
	$K(A)$    & $\frac{\|r\|}{\|b\|}$ & $\frac{\|x-\tilde{x}\|}{\|\tilde{x}\|}$ \\
	\hline
	nero    & 0           & zero                                          \\
	marrone & 1           & uno                                           \\
	\hline
\end{tabular}
\end{center}