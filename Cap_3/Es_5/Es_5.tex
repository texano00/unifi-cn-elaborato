Il seguente codice MatLab, contiene la chiamata delle due funzioni descritte negli esercizi precedenti, (\textit{risolutoreLDLt}  dell'es. 3.3 e risolutoreLUpiv dell'es. 3.4) per dimostrarne l'utilizzo tramite alcuni esempi:\\\
\lstinputlisting[language=Matlab]{Cap_3/Es_5/Es_5.m}
\begin{description}
\item \textbf{Esempio :} \textit{$risolutoreLDL^t$}\\\\
	con i seguenti parametri di input :\\\	 
	\[
	A =\begin{bmatrix}
		1 & 3   & 4    \\ 
		3 & 1   & -60  \\
		4 & -60 & -640 
	\end{bmatrix}
	\]\\
	\[
	x =\begin{bmatrix}
		1 \\
		2 \\
		3 
	\end{bmatrix}
	\]\\
	Dove $A = LDL^t$\\	 
	\[
	L =\begin{bmatrix}
		1 & 0 & 0 \\ 
		3 & 1 & 0 \\
		4 & 9 & 1 
	\end{bmatrix}
	\]\\ 
	\[
	D =\begin{bmatrix}
		1 & 0  & 0  \\ 
		0 & -8 & 0  \\
		0 & 0  & -8 
	\end{bmatrix}
	\]\\ 
	\[
	L^t =\begin{bmatrix}
		1 & 3 & 4 \\ 
		0 & 1 & 9 \\
		0 & 0 & 1 
	\end{bmatrix}
	\]\\\	
	Il risultato ottenuto è:\\
	\[
	x =\begin{bmatrix}
		-22.3750000000000 \\
		9.12500000000000  \\
		-1                
	\end{bmatrix}
	\]\\\\
\item \textbf{Esempio :} \textit{$risolutoreLUP$}\\\\	
	con i seguenti parametri di input :\\\	  
	\[
	A =\begin{bmatrix}
		2 & 4 & 9 \\ 
		6 & 5 & 6 \\
		8 & 2 & 9 
	\end{bmatrix}
	\]\\
	\[
	xx =\begin{bmatrix}
		1 \\
		2 \\
		3 
	\end{bmatrix}
	\]\\
	Dove $A = LU$\\ 
	\[
	L =\begin{bmatrix}
		1 & 0 & 0 \\ 
		6 & 1 & 0 \\
		8 & 2 & 1 
	\end{bmatrix}
	\]\\ 
	\[
	U =\begin{bmatrix}
		2 & 4 & 9 \\ 
		0 & 5 & 6 \\
		0 & 0 & 9 
	\end{bmatrix}
	\]\\ 
	\[
	P =\begin{bmatrix}
		1 & 0 & 0 \\ 
		0 & 1 & 0 \\
		0 & 0 & 1 
	\end{bmatrix}
	\]\\\
	Il risultato ottenuto è:\\
	\[
	xx =\begin{bmatrix}
		1.40000000000000  \\
		-1.20000000000000 \\
		0.333333333333333 
	\end{bmatrix}
	\]\\\\
\end{description}
La tabella sottostante contiene, per ogni esempio considerato, il numero di condizionamento di \textit{A}, in \textit{norma 2}, con l'utilizzo del comando \textit{cond} di Matlab :\\\
\begin{center}
\begin{tabular}{ | l | c | r | }
	\hline
	$K(A)$    & $\frac{\|r\|}{\|b\|}$ & $\frac{\|x-\tilde{x}\|}{\|\tilde{x}\|}$ \\
	\hline
	nero    & 0           & zero                                          \\
	marrone & 1           & uno                                           \\
	rosso   & 2           & due                                           \\
	arancio & 3           & tre                                           \\
	giallo  & 4           & quattro                                       \\
	verde   & 5           & cinque                                        \\
	blu     & 6           & sei                                           \\
	viola   & 7           & sette                                         \\
	\hline
\end{tabular}
\end{center}