Il seguente codice MatLab, contiene la chiamata delle due funzioni descritte negli esercizi precedenti, (\textit{risolutoreLDLt}  dell'es. 3.3 e risolutoreLUpiv dell'es. 3.4) per dimostrarne l'utilizzo tramite alcuni esempi:\\\
	\lstinputlisting[language=Matlab]{Cap_3/Es_5/Es_5.m}
\begin{description}
	\item \textbf{Esempio :} \textit{$risolutoreLDL^t$}\\\\
		con i seguenti parametri di input :\\\	 
		\[
		A_1 = LDL^t =\begin{bmatrix}
			3 & 2 & -1 \\ 
			2 & 7 &  7 \\
			-1 & 7 & 30 
		\end{bmatrix} \quad
		\hat{x_1} =\begin{bmatrix}
			4 \\
			5 \\
			3                
		\end{bmatrix} \quad
		b_1 = LDL^t \hat{x_1} =\begin{bmatrix}
			19 \\
			64 \\
			121                
		\end{bmatrix}
		\]\\
		\[
		L =\begin{bmatrix}
			1 		& 0 	 & 0 \\ 
			0,6667  & 1 	 & 0 \\
	   		-0,3333 & 1.3529 & 1 
		\end{bmatrix} \quad
		D =\begin{bmatrix}
			3 & 0      & 0 	     \\ 
			0 & 5.6667 & 0       \\
			0 & 0      & 19.2941 
		\end{bmatrix} \quad
		L^t =\begin{bmatrix}
			1 & 0.6667 & -0.3333 \\ 
			0 & 1      &  1.3529 \\
			0 & 0      &  1 
		\end{bmatrix}
		\]\\\\
	Il risultato ottenuto è:\\
		\[
		x_1 =\begin{bmatrix}
			4 \\
			5 \\
			3                
		\end{bmatrix} \quad
		r_1 = LDL^t*x_1-b_1 =\begin{bmatrix}
			0 \\
		  	-7.1054e-15 \\
		  	0
		\end{bmatrix}
		\]\\\\
\item \textbf{Esempio :} \textit{risolutoreLUpiv}\\\\	
	con i seguenti parametri di input :\\\	  
		\[
		A_2 = LU =\begin{bmatrix}
			-23 & 5  & -21 & 8 \\ 
			0 	& 0  & 5   & 7 \\
			1 	& 54 & 7   & 9 \\
			0 	& -8 & 12  & 4  
		\end{bmatrix} \quad
		\hat{x_2} =\begin{bmatrix}
			2 \\
			8 \\
			3 \\
			5               
		\end{bmatrix} \quad
		b_2 = LDL^t \hat{x_2} =\begin{bmatrix}
			-29 \\
			50  \\
			500 \\
			-8                
		\end{bmatrix}
		\]\\
		\[
		L =\begin{bmatrix}
			1   	&  0  		& 0 	 & 0 \\ 
			-0.0435 &  1  		& 0 	 & 0 \\
		 	0   	&  -0.1476  & 1 	 & 0 \\
		 	0   	&  0  		& 0.3877 & 1 		    
		\end{bmatrix} \quad
		U =\begin{bmatrix}
			-23 & 5 	  & -21		& 8   	   \\ 
			0 	& 54.2174 & 6.0870 	& 9.3478   \\ 
			0 	& 0 	  & 12.8982 & 5.3793   \\ 
			0 	& 0 	  & 0 		& 4.92147		
		\end{bmatrix}
		\]\\ 
		\[
		P =\begin{bmatrix}
			1 & 0 & 0 & 0 \\ 
			0 & 0 & 1 & 0 \\
			0 & 0 & 0 & 1 \\
			0 & 1 & 0 & 0 
		\end{bmatrix} \quad
		Pb =\begin{bmatrix}
			-29 \\
			500 \\
			-8  \\
			50               
		\end{bmatrix}
		\]\\\\
	Il risultato ottenuto è:\\
		\[
		x_2 =\begin{bmatrix}
			2 \\
		  	8 \\
		  	3 \\
		  	5 
		\end{bmatrix} \quad
		r_2 = LU*x_2-b_2 =\begin{bmatrix}
			7.1054e-15 \\
		  	7.1054e-15 \\
		  	0 \\
		  	-3.5527e-15 
		\end{bmatrix}
		\]\\\\
\end{description}
La tabella sottostante contiene, per ogni esempio considerato, il numero di condizionamento di \textit{A}, in \textit{norma 2}, con l'utilizzo del comando \textit{cond} e \textit{norm} di Matlab :\\\
\begin{center}
	\begin{tabular}{ | c | c | c | c | }
		\hline
			\textit{A} & $K_2(A)$ & $\frac{\|r\|}{\|b\|}=krb$ & $\frac{\|x-\tilde{x}\|}{\|\tilde{x}\|}=kxtx$ \\
		\hline
			$A_1$ & $k_1=20.0572$ & $krb_1=5.1416e-17$ & $kxtx_1=1.4043e-16$ \\
			$A_2$ &	$k_2=17.6716$ & $krb_2=2.1173e-17$ & $kxtx_2=1.0771e-16$ \\
		\hline
	\end{tabular}
\end{center}