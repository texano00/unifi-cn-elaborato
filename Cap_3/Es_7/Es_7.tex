Il seguente codice MatLab, contiene l'implementazione di una funzione per la risoluzione di un sistema lineare $Ax = b$ con la seguente tipologia di matrice \textit{A} :\\\
	\[
	A = \begin{bmatrix}
		1 & 0 & \cdots & \cdots & 0 \\
		\alpha & 1 & 0 & \cdots & 0 \\
		0 & \ddots & \ddots  & \ddots & \vdots \\
		\vdots & \ddots & \ddots & \ddots & \vdots \\
		0 & \cdots & \cdots & \alpha & 1
  	\end{bmatrix}
  	\]
\begin{itemize}
	\item \textbf{Metodo matrice triangolare inferiore modificato}
		\lstinputlisting[language=Matlab]{Cap_3/Es_7/triangolareInferioreMod.m}
	\item \textbf{Implementazione}\\\
		Il seguente codice MatLab contiene la chiamata della funzione precedentemente definita con i rispettivi valori di input(con $n=12$, $A \in \mathbb{R}^{12X12}$, $b_1 \in \mathbb{R}^{12}$ e $b_2 \in \mathbb{R}^{12}$):
		\[
		A = \begin{bmatrix}
			1 & 0 & \cdots & \cdots & 0 \\
			100 & 1 & 0 & \cdots & 0 \\
			0 & \ddots & \ddots  & \ddots & \vdots \\
			\vdots & \ddots & \ddots & \ddots & \vdots \\
			0 & \cdots & \cdots & 100 & 1
			\end{bmatrix} \quad
		b_1 = \begin{bmatrix}
			1 \\
 			101 \\
 	 		\vdots \\
 	 		\vdots \\
	  		101 
  		\end{bmatrix} \quad
  		b_2 = 0.1*\begin{bmatrix}
			1 \\
 			101 \\
 	 		\vdots \\
	 		\vdots \\
	  		101 
  		\end{bmatrix} 
  		\]\\\
  		\lstinputlisting[language=Matlab]{Cap_3/Es_7/Es_7.m}
  		restituendo i seguenti valori:
  		\[
  		x_1 = \begin{bmatrix}
  			1   \\
  			\vdots  \\
  			1
  		\end{bmatrix} \quad
  		x_2 = \begin{bmatrix}
  		    1.000000000000000e-01 \\
  			1.000000000000014e-01 \\
  			9.999999999985931e-02 \\
  			1.000000000140702e-01 \\
  			9.999999859298470e-02 \\
     		1.000001407015318e-01 \\
     		9.998592984681665e-02 \\
     		1.014070153183368e-01 \\
    		-4.070153183368319e-02 \\
    		1.417015318336832e+01 \\
    		-1.406915318336832e+03 \\
     		1.407016318336832e+05 \\
     		-1.407015308336832e+07
  		\end{bmatrix}
  		\]
	\begin{itemize}
		\item \textbf{Studio condizionamento}\\\
			Risulta
				\[
					\|A\|_\infty = \underset{i=1,...,n}{max} \sum_{j=1}^n|a_{i,j}|=101
				\]
				\[
					\|A^{-1}\|_\infty = \underset{i=1,...,n}{max} \sum_{j=1}^n|a_{i,j}|=\sum_{j=1}^n|a_{n,j}|=\sum_{s=0}^{n-1}10^{2s}=\frac{10^{2n}-1}{10^2-1}=\frac{10^{2n}-1}{99}
				\]
				\[
					quindi 
					\quad
					k_\infty(A) = \|A\|_\infty \|A^{-1}\|_\infty=101\frac{10^{2n}-1}{99}>10^{2n}
				\]\\\
			nel caso $n=12$, si ha $k_\infty(A)>10^{24}$ quindi il problema è malcondizionato. Su tale matrice, la function \textit{cond} restiruisce \textit{Inf}.
			La \textit{norma} $\infty$ su una matrice è la somma delle righe, la \textit{norma} $1$ è la somma massima delle colonne; nella matrice A tutte le colonne, come tutte le righe, hanno somma 101 quindi $\|A\|_\infty=\|A\|_1=101$. Nella matrice $A^{-1}$, la \textit{norma} $\infty$ considera l'\textit{n}-esima riga mentre la \textit{norma} $1$ la prima colonna, in ogni caso, $\|A\|_\infty=\|A\|_1=\frac{10^{24}-1}{99}$. Quindi $k_\infty(A)>10^{24}$.\\\
			Considerando il vettore $\underline{b2}$ come una perturbazione di $\underline{b1}$ si ha:
			\[
				\Delta\underline{b1}=\underline{b2}-\underline{b1}=\begin{bmatrix}
				-0.9 \\
				-90.9 \\
				\vdots \\
				-90.9
				\end{bmatrix}
			\]\\\
			segue
			\[
				\frac{\|\Delta\underline{b1}\|}{\|\underline{b1}\|}\approx \frac{\sqrt{0.9+9*90.9^2}}{\sqrt{1+9*101^2}}\approx 1. 
			\]\\\
			Quindi:
			\[
				\frac{\|\Delta\underline{x}\|}{\|\underline{x}\|}\leq k(A)\Bigg(\frac{\|\Delta\underline{b1}\|}{\|\underline{b1}\|}+\frac{\|\Delta A\|}{\|{A}\|}\Bigg) = k(A)\frac{\|\Delta\underline{b1}\|}{\|\underline{b1}\|}\approx 10^{24}
			\]\\\
			ovvero a fronte di una perturbazione del vettore \underline{b1} di $0.1$, si ha un errore sul risultato dell'ordine di $10^{24}$.
	\end{itemize}
\end{itemize}