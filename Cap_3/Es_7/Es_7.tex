Il seguente codice MatLab, contiene l'implementazione di una funzione per la risoluzione di un sistema lineare $Ax = b$ con la seguenti tipologia di matrice \textit{A} :
	\[
	A = \begin{bmatrix}
		1     		& 0      	& 0      	& 0  	 & 0 & \cdots \\ 
    	\alpha 		& 1      	& 0      	& 0  	 & 0 & \cdots \\
    	0      		& \alpha 	& 1      	& 0  	 & 0 & \cdots \\
    	0      		& 0      	& \alpha 	& 1  	 & 0 & \cdots \\
    	\cdots     	& \cdots    & \cdots    & \alpha & 1 & \cdots 
  	\end{bmatrix}
  	\]
\begin{itemize}
	\item \textbf{Metodo matrice triangolare inferiore modificato}
		\lstinputlisting[language=Matlab]{Cap_3/Es_7/triangolareInferioreMod.m}
	\item \textbf{Esempio}\\\
		Il seguente codice MatLab contiene la chiamata della funzione precedentemente definita con i rispettivi valori di input:
		\[
		A = \begin{bmatrix}
			1 & 0 & 0 & 0 & 0 \\
			5 & 1 & 0 & 0 & 0 \\
			0 & 5 & 1 & 0 & 0 \\
			0 & 0 & 5 & 1 & 0 \\
			0 & 0 & 0 & 5 & 1
		\end{bmatrix} \quad
		b = \begin{bmatrix}
			1 \\
 			2 \\
 	 		3 \\
	  		4 \\
  			5
  		\end{bmatrix} 
  		\]\\\
  		\lstinputlisting[language=Matlab]{Cap_3/Es_7/Es_7.m}
  		restituendo il seguente valore:
  		\[
  		b = \begin{bmatrix}
  			1   \\
  			-3  \\
  			18  \\
  			-86 \\
  			435
  		\end{bmatrix} 
  		\]
\end{itemize}