Per la risoluzione di un sistema lineare $Ax=b$ con $A=LDL^t$ , viene chiamato il seguente codice in MatLab:\\\
\begin{itemize}
\item \textbf{Metodo risoluzione $LDL^tx=b$}
\lstinputlisting[language=Matlab]{Cap_3/Es_3/risolutoreLDLt.m}
\end{itemize}
il quale implementa in ordine:
\begin{enumerate}
\item
Una funzione di fattorizzazione di un matrice $LDL^t$ passata come input e restituisce una matrice \textit{A} riscritta con le informazioni di \textit{L}, \textit{D} e $L^t$ (guarda es. 3.2).\\\
\item
Una funzione per il calcolo del vettore incognite $x_1$ di una matrice triangolare inferiore a diagonale unitaria \textit{A} (con l'utilizzo dei comandi \textit{tril(A,k<0)} che restituisce gli elementi sotto la k-esima diagonale di A; \textit{eye(n)} che restituisce una matrice di identità (tutti i valori zeri a parte i termini diagonali) di grandezza \textit{n}) passata come input insieme al vettore dei termini noti \textit{b} del sistema (guarda es. 3.1). \\\
\item
Una funzione per il calcolo del vettore incognite $x_2$ di una matrice diagonale \textit{A} (con l'utilizzo del comando \textit{diag(A)} che restituisce un vettore colonna degli elementi diagonali principali di A) passata come input insieme al vettore dei termini noti $x_1$:\\\
\begin{itemize}
\item \textbf{Metodo 'vettore' diagonale}
\lstinputlisting[language=Matlab]{Cap_3/Es_3/diagonale.m}
\end{itemize}
\item
Una funzione per il calcolo del vettore incognite finale $x$ del sistema lineare di una matrice triangolare superiore a diagonale unitaria \textit{A} (con l'utilizzo dei comandi \textit{tril(A,k>0)} che restituisce gli elementi sopra la k-esima diagonale di A; \textit{eye(n)} che restituisce una matrice di identità (tutti i valori zeri a parte i termini diagonali) di grandezza \textit{n}) passata come input insieme al vettore dei termini noti $x_2$:\\\
\begin{itemize}
\item \textbf{Metodo matrice triangolare superiore}
\lstinputlisting[language=Matlab]{Cap_3/Es_3/triangolareSuperiore.m}
\end{itemize}
\end{enumerate}
Come descritto nei codici, a scapito di eleganza e leggibilità del codice stesso, si è deciso di non estrarre da $LDL^t$ i fattori \textit{L}, \textit{D} e $L^t$ in modo esplicito, così da non occupare inutilmente posizioni di memoria che si erano risparmiate riscrivendo $LDL^t$ con \textit{L}, \textit{D} e $L^t$.