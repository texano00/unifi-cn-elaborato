Per la risoluzione di un sistema lineare $Ax=b$ con $A=LDL^t$ , viene chiamato il seguente codice in MatLab:\\\
\lstinputlisting[language=Matlab]{Cap_3/Es_3/risolutoreLDLt.m}
il quale implementa in ordine:
\begin{enumerate}
\item
una funzione di scomposizione di un matrice $LDL^t$ passata come input e restituisce una matrice triangolare inferiore a diagonale unitaria \textit{L}, una matrice diagonale \textit{D} e una matrice triangolare superiore a diagonale unitaria $L^t$:\\\
\lstinputlisting[language=Matlab]{Cap_3/Es_3/scomponiLDLt.m}
\item
una funzione per il calcolo del vettore incognite $x_1$ di una matrice triangolare inferiore a diagonale unitaria \textit{L} passata come input insieme al vettore dei termini noti \textit{b} del sistema (guarda esercizio 3.1). \\\
\item
una funzione per il calcolo del vettore incognite $x_2$ di una matrice diagonale \textit{D} passata come input insieme al vettore dei termini noti $b=x_1$:\\\
\lstinputlisting[language=Matlab]{Cap_3/Es_3/diagonale.m}
\item
una funzione per il calcolo del vettore incognite finale $x$ del sistema lineare di una matrice triangolare superiore a diagonale unitaria $L^t$ passata come input insieme al vettore dei termini noti $b=x_2$:\\\
\lstinputlisting[language=Matlab]{Cap_3/Es_3/triangolareSuperiore.m}
\end{enumerate}