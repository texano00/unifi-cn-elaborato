Per la risoluzione di un sistema lineare $Ax=b$ con $A=LDL^t$ , viene chiamato il seguente codice in MatLab:\\\
\begin{itemize}
\item \textbf{Metodo risoluzione $LDL^tx=b$}
\lstinputlisting[language=Matlab]{Cap_3/Es_3/risolutoreLDLt.m}
il quale implementa in ordine:
\begin{enumerate}
\item \textbf{fattorizzazioneLDLt($LDL^t$)}\\
Una funzione di fattorizzazione di un matrice $LDL^t$ passata come input e restituisce una matrice \textit{A} riscritta con le informazioni di \textit{L}, \textit{D} e $L^t$ (guarda es. 3.2).\\\
\item \textbf{triangolareInferiore(L,b)}\\
Una funzione per il calcolo del vettore incognite $x_1$ di una matrice triangolare inferiore a diagonale unitaria \textit{L} (con l'utilizzo dei comandi \textit{tril($LDL^t$,k<0)} che restituisce gli elementi sotto la k-esima diagonale di $LDL^t$; \textit{eye(n)} che restituisce una matrice di identità (tutti i valori zeri a parte i termini diagonali) di grandezza \textit{n}) passata come input insieme al vettore dei termini noti \textit{b} del sistema (guarda es. 3.1). \\\
\item \textbf{diagonale(D,$x_1$)}\\
Una funzione per il calcolo del vettore incognite $x_2$ di una matrice diagonale \textit{D} (con l'utilizzo del comando \textit{diag(A)} due volte, in quando la prima mi restituisce un vettore colonna degli elementi diagonali principali di $LDL^t$ e la seconda una matrice diagonale con tutti e solo gli elementi della diagonale $\neq 0$ ) passata come input insieme al vettore dei termini noti $x_1$:\\\
\begin{itemize}
\item \textbf{Metodo 'vettore' diagonale}
\lstinputlisting[language=Matlab]{Cap_3/Es_3/diagonale.m}
\end{itemize}
\item \textbf{triangolareSuperiore($L^t$,b)}\\
Una funzione per il calcolo del vettore incognite finale $x$ del sistema lineare di una matrice triangolare superiore a diagonale unitaria \textit{$L^t$} (con l'utilizzo dei comandi \textit{tril($LDL^t$,k<0)} che restituisce gli elementi sotto la k-esima diagonale di $LDL^t$; \textit{eye(n)} che restituisce una matrice di identità (tutti i valori zeri a parte i termini diagonali) di grandezza \textit{n}; al tutto viene aggiunta un \textit{apice " ' "} per calcolarne la trasposta) passata come input insieme al vettore dei termini noti $x_2$:\\\
\begin{itemize}
\item \textbf{Metodo matrice triangolare superiore}
\lstinputlisting[language=Matlab]{Cap_3/Es_3/triangolareSuperiore.m}
\end{itemize}
\end{enumerate}
\end{itemize}