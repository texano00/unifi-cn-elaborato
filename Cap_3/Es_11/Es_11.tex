Il seguente codice effettua la chiamata della funzione \textbf{newtonNonLin}, partendo dalla funzione 
$f = f(x_1,x_2)=x_1^2+x_2^3-x_1x_2$, per risolvere il seguente sistema nonlineare con i relativi parametri di input:
	\[
	F(x)=0 \quad
	F=\begin{bmatrix}
		\frac{\partial f}{\partial x_1} \\
		\frac{\partial f}{\partial x_2} 
	\end{bmatrix}= 
	\begin{bmatrix}
		2x_1-x_2 \\
		3x_2^2-x_1
	\end{bmatrix}=f \quad
	\textit{con punto di innesco } x_1=\frac{1}{2} \quad x_2=\frac{1}{2}
	\]
	\[
	J_F=\begin{bmatrix}
		2-x_2 & 2x_1-1 \\
		3x_2^2-1 & 6x_2-x_1 
	\end{bmatrix}
	\]
	\[
	imax=100, \quad 
	tolx=10^{-t} \quad t=\{3,6\}
	\]
\lstinputlisting[language=Matlab]{Cap_3/Es_11/Es_11.m}
Qui di seguito è riportata una tabella con le seguenti informazioni ($i$ numero di iterazioni eseguite, $\|n\|$ norma euclidea dell'ultimo incremento e $\|e\|$ norma euclidea dell'errore con cui viene approssimato il risultato esatto):
\begin{center}
	\begin{tabular}{|c|c|c|c|}
		\hline
			$tol_x=10^{-t}$ & $i$ & $\|n\|$ & $\|e\|$ \\
		\hline
    		$10^{-3}$ & $i = 17$ & $\|n_1\| = 0.190126478566088$ & $\|e_1\| = $0.003794501517081 \\
    		$10^{-6}$ & $i = 51$ & $\|n_2\| = 0.186343470827848$ & $\|e_2\| = $4.480819013409465e-06 \\
		\hline
	\end{tabular}
\end{center}