Il seguente codice MatLab, contiene la chiamata della funzione descritta nell'esercizio precedente, (\textit{risolutoreQR}  dell'es. 3.8) e del comando $A \backslash b$, per dimostrarne l'utilizzo tramite alcuni esempi:\\\
	\lstinputlisting[language=Matlab]{Cap_3/Es_9/Es_9.m}
\textbf{Esempio}
\begin{enumerate} 
	\item con input:
		\[
		A_1 =\begin{bmatrix}
			3 & 2 & 1 \\
			1 & 2 & 3 \\
			1 & 2 & 1 \\
			2 & 1 & 2 
		\end{bmatrix} \quad
		b_1 =\begin{bmatrix}
			10 \\
			10 \\
			10 \\
			10
		\end{bmatrix}
		\]\\\
	\begin{itemize}
		\item \textbf{Fattorizzazione QR e $A \backslash b$}
			\[
			x_{1(QR)} =\begin{bmatrix}
				1.4 \\
				2.8 \\
				1.4
			\end{bmatrix} \quad
			x_{1(A \backslash b)} =\begin{bmatrix}
				1.4 \\
				2.8 \\
				1.4
			\end{bmatrix}
			\]\\\	
	\end{itemize}
	\item con input:
		\[
		A_2 =\begin{bmatrix}
			9  & -14 & -3 \\
			4  & 9	 & 6  \\
			33 & 4   & 12 \\
			7  & -23 & 4 
		\end{bmatrix} \quad
		b_2 =\begin{bmatrix}
			12  \\
			-5  \\
			9   \\
			-25
		\end{bmatrix}
		\]\\\
	\begin{itemize}
		\item \textbf{Fattorizzazione QR e $A \backslash b$}
			\[
			x_{2(QR)} =\begin{bmatrix}
				1.4455  \\
				0.9138  \\
				-3.4834
			\end{bmatrix} \quad
			x_{2(A \backslash b)} =\begin{bmatrix}
				1.4455  \\
				0.9138  \\
				-3.4834
			\end{bmatrix}
			\]	
	\end{itemize}
\end{enumerate}