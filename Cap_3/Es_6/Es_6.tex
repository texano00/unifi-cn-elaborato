Il seguente codice MatLab, contiene l'implementazione di una funzione per la \textit{fattorizzazione LU} di una matrice \textit{A}:\\\
\begin{itemize}
	\item \textbf{Metodo fattorizzazione \textit{LU}}	
		\lstinputlisting[language=Matlab]{Cap_3/Es_6/fattorizzazioneLU.m}
\end{itemize}
Il seguente codice Matlab, contiene la chiamata della funzione di \textit{fattorizzazione LU} (\textit{L} viene ricavata con l'utilizzo dei comandi \textit{tril(A,k<0)} che restituisce gli elementi sotto la k-esima diagonale di A; \textit{eye(n)} che restituisce una matrice di identità (tutti i valori zeri a parte i termini diagonali) di grandezza \textit{n}), (\textit{U} viene ricavata con l'utilizzo del comando \textit{tril(A)} che restituisce la parte triangolare superiore di A) e successivamente vengono eseguiti i comandi $U \backslash (L \backslash b)$ \textit{Gauss senza pivoting} e $A \backslash b$ \textit{Gauss con pivoting}:\\\
	\lstinputlisting[language=Matlab]{Cap_3/Es_6/Es_6.m}
restituendo rispettivamente:\\\
\begin{enumerate}
	\item 
	\begin{itemize}
		\item \textbf{Fattorizzazione LU}
			\[
			A =\begin{bmatrix}
				10^{(-13)} &   1  \\
	 			1000   & -999 \\
			\end{bmatrix}
			\]\\
			\[
			L*U =\begin{bmatrix}
				1    & 0 \\
				10^{(13)} & 1 \\
			\end{bmatrix} *
			\begin{bmatrix}
				10^{(-13)} & 1    \\
				0      	  & -10^{(13)} \\
			\end{bmatrix} 
			= \begin{bmatrix}
				10^{(-13)} &   1  \\
				1   & 1 \\
			\end{bmatrix} = LU
			\]\\\
	\end{itemize}
	\item
		\[
		e = \begin{bmatrix}
			1 \\
			1
		\end{bmatrix} \quad 
		b = Ae =\begin{bmatrix}
			1.000000000000100 \\
			2                 
		\end{bmatrix}
		\]\\\
	\begin{itemize}
	\item \textbf{Gauss senza pivoting}
		\[
		Sp = U \backslash (L \backslash b) = \begin{bmatrix}
			0.999200722162641 \\
    		1
		\end{bmatrix}
		\]\\\
	\end{itemize}
	\begin{itemize}
	\item \textbf{Gauss con pivoting}
		\[
		Cp = A \backslash b =\begin{bmatrix}
			1.000000000000000 \\
    		1.000000000000000 
		\end{bmatrix}
		\]
	\end{itemize}
\end{enumerate}