Il seguente codice MatLab contiene la soluzione del problema dell'Es.5 :\\\
	\lstinputlisting[language=Matlab]{Cap_5/Es_5/Es_5.m}
restituendo i seguenti valori:\\\
\begin{itemize}
	\item
		\textbf{Formula dei Trapezi Composita}:\\\
			Per trovare \textit{n (sottointervalli)} tale che l'\textit{errore} commesso sia minore della tol fornita usando la \textit{Trapezi Composita}, abbiamo \textit{iterato n} partendo da $1000000$ fino ad un massimo di $10000000$ con passi di $1000000$.
			Con questo metodo \textit{a tentativi} abbiamo trovato che la \textit{tolleranza} è soddisfatta quando $9000000<n<10000000$.\\\
			\begin{center}
				\textbf{Ultime due iterazioni:}\\\
			\begin{tabular}{|c|c|c|c|}
				\hline
					$tol$ & $num. val. funz. = n \quad (sottointervalli)$ & $I=tc$ & $E_1^{(n)}$ \\
					\hline
						$10^{-9}$ & $9000000$ & $1.001028594957969e-06$ & $1.028594957968679e-09$ \\
						$10^{-9}$ & $10000000$ & $1.000833194477503e-06$ & $8.331944775031580e-10$ \\
					\hline
			\end{tabular}
			\end{center}
	\item
		\textbf{Formula dei Simpson Composita}:\\\
			Per cercare di trovare \textit{n (sottointervalli)} tale che l'\textit{errore} commesso sia minore della tol fornita usando la \textit{Simpson Composita}, abbiamo \textit{iterato n} partendo da $1000000$ fino ad un massimo di $10000000$ con passi di $1000000$ non riuscendo però a raggiungere la tolleranza richiesta.
			Con questo metodo \textit{a tentativi} abbiamo trovato una limitazione inferiore del numero dei sottointervalli richiesti per soddisfare la tolleranza, ossia $n>10000000$.\\\
			\begin{center}
				\textbf{Ultime due iterazioni:}\\\
			\begin{tabular}{|c|c|c|c|}
				\hline
					$tol$ & $num. val. funz. = n \quad (sottointervalli)$ & $I=sc$ & $E_2^{(n)}$ \\
					\hline
						$10^{-9}$ & $9000000$ & $2.000000105805421e-06$ & $1.000000105805421e-06$ \\
						$10^{-9}$ & $10000000$ & $2.000000069423777e-06$ & $1.000000069423778e-06$ \\
					\hline
			\end{tabular}
			\end{center}
	\item
		\textbf{Formula dei Trapezi Adattiva}:\\\
			In questo caso invece siamo riusciti a trovare esattamente il numero di valutazioni necessarie per raggiungere la tollerenza richiesta.
			\begin{center}
			\begin{tabular}{|c|c|c|}
				\hline
					$tol$ & $num. val. funz.$ & $I=ta$ \\
    				\hline
    					$10^{-9}$ & $25943$ & $1.000000011252939e-06$ \\
				\hline
			\end{tabular}
			\end{center}
	\item
		\textbf{Formula di Simpson Adattiva}:\\\
			Come nel caso precedente, siamo riusciti a trovare esattamente il numero di valutazioni necessarie per raggiungere la tollerenza richiesta.
			\begin{center}
			\begin{tabular}{|c|c|c|}
				\hline
					$tol$ & $num. val. funz.$ & $I=sa$ \\
    				\hline
    					$10^{-9}$ & $349$ & $1.000000016469981e-06$ \\
					\hline
			\end{tabular}
			\end{center}
\end{itemize}